\documentclass[a4paper,10pt]{article}

\usepackage[utf8]{inputenc}
\usepackage[francais]{babel}
\usepackage[T1]{fontenc}
\usepackage{mathpazo} %http://www.ctan.org/tex-archive/fonts/mathpazo
\usepackage{stmaryrd} %http://www.ctan.org/pkg/stmaryrd
\usepackage{amsmath} %http://www.ctan.org/pkg/amsmath
\usepackage{amssymb}
\usepackage{mathrsfs}

\usepackage{amsthm} %http://www.ctan.org/pkg/amsthm
\usepackage{proof}

\usepackage[colorlinks=true]{hyperref} %http://www.ctan.org/tex-archive/macros/latex/contrib/hyperref/
\hypersetup{urlcolor=black,linkcolor=black}

\usepackage{footmisc} %http://www.ctan.org/tex-archive/macros/latex/contrib/footmisc

\usepackage{enumerate}
\usepackage{ulem} %http://www.ctan.org/tex-archive/macros/latex/contrib/ulem
\normalem
\usepackage{cancel} %http://www.ctan.org/tex-archive/macros/latex/contrib/cancel

\usepackage{fullpage} %http://www.ctan.org/tex-archive/macros/latex/contrib/preprint/
\setlength{\parindent}{0pt}
\setlength{\parskip}{\medskipamount}

\usepackage{pgffor}
\usepackage{tikz}
\usetikzlibrary{arrows,shapes.arrows, chains, positioning, automata, graphs}
\usepackage{graphviz}

\usepackage[ruled,vlined,english]{algorithm2e}
\providecommand{\SetAlgoLined}{\SetLine}
\providecommand{\DontPrintSemicolon}{\dontprintsemicolon}

\usepackage{forest}
\usepackage{comment} %http://www.ctan.org/tex-archive/macros/latex/contrib/comment
\usepackage{multirow} %http://www.ctan.org/tex-archive/macros/latex/contrib/multirow
\usepackage{diagbox} %http://www.ctan.org/tex-archive/macros/latex/contrib/diagbox

\usepackage{textcomp} %http://www.ctan.org/pkg/textcomp

\usepackage{listings} %http://www.ctan.org/tex-archive/macros/latex/contrib/listings/
\lstset{numbers=left,language=Caml}

\newcounter{ThComp}
\newcounter{DefComp}

\newtheorem*{fact}{Fact}
\newtheorem*{csq}{Consequence}
\newtheorem{thm}[ThComp]{Theorem}
\newtheorem{theorem}[ThComp]{Theorem}
\newtheorem{propo}[ThComp]{Proposition}
\newtheorem{proposition}[ThComp]{Proposition}
\newtheorem{lemma}[ThComp]{Lemma}
\newtheorem*{corol}{Corollary}
\newtheorem{prop}[ThComp]{Property}
\newtheorem{property}[ThComp]{Property}
\theoremstyle{definition}
\newtheorem*{ex}{Example}
\newtheorem*{exs}{Examples}
\newtheorem{exo}{Exercise}
\newtheorem{defi}[DefComp]{Definition}
\newtheorem*{notation}{Notation}
\newtheorem{definition}[DefComp]{Definition}
\newtheorem{algo}{Algorithm}
\theoremstyle{remark}
\newtheorem*{Rq}{Remark}
\newcommand{\ra}{\rightarrow}
\newcommand{\la}{\leftarrow}


\newcommand{\RR}{\mathbb{R}}
\newcommand{\ZZ}{\mathbb{Z}}
\newcommand{\NN}{\mathbb{N}}
\newcommand{\PP}{\mathbb{P}}
\newcommand{\EE}{\mathbb{E}}
\newcommand{\IE}{\mathbb{E}}
\newcommand{\IR}{\mathbb{R}}
\newcommand{\IZ}{\mathbb{Z}}
\newcommand{\IN}{\mathbb{N}}
\newcommand{\IP}{\mathbb{P}}

\newcommand{\cF}{\mathcal{F}}
\newcommand{\ck}{\mathcal{K}}
\newcommand{\cL}{\mathcal{L}}
\newcommand{\cN}{\mathcal{N}}
\newcommand{\cNU}{\mathcal{NU}}
\newcommand{\A}{\mathcal{A}}
\newcommand{\B}{\mathcal{B}}
\newcommand{\F}{\mathcal{F}}
\renewcommand{\L}{\mathcal{L}}
\newcommand{\N}{\mathcal{N}}

\newcommand{\ens}[1]{\left\{ #1 \right\}}
\newcommand{\set}[1]{\left\{ #1 \right\}}
\renewcommand{\leq}{\leqslant}
\renewcommand{\geq}{\geqslant}
\renewcommand{\le}{\leqslant}
\renewcommand{\ge}{\geqslant}
\newcommand{\cplx}[1]{\mathcal O \left( #1 \right)}
\newcommand{\floor}[1]{\left \lfloor #1 \right \rfloor}
\newcommand{\ceil}[1]{\left\lceil #1 \right\rceil}
\newcommand{\brackets}[1]{\left\llbracket #1 \right\rrbracket}
\newcommand{\donne}{\rightarrow}
\newcommand{\gives}{\rightarrow}
\newcommand{\dans}{\to}
\newcommand{\booleen}{\set{0,1}^*}
\newcommand{\eps}{\varepsilon}
\renewcommand{\implies}{~\Rightarrow~}
\newcommand{\tildarrow}{\rightsquigarrow}
\newcommand{\blank}{\texttt{\char32}}
\newcommand{\trans}[1]{\xrightarrow{#1}}
\newcommand{\rules}[1]{\xrightarrow{#1}}
\newcommand{\todo}[1]{\Large\textcolor{red}{#1}\normalsize}
\newcommand{\argmin}{\text{argmin}}
\newcommand{\rainbowdash}{\vdash}
\newcommand{\notrainbowdash}{\nvdash}
\newcommand{\rainbowDash}{\vDash}
\newcommand{\notrainbowDash}{\nvDash}
\newcommand{\Rainbowdash}{\Vdash}
\newcommand{\notRainbowdash}{\nVdash}
\newcommand{\bottom}{\bot}

%TD/TP
\newenvironment{answer}{\color{blue}}{}


%EvalPerf
\newcommand{\Var}{\text{Var}}
\newcommand{\prob}[1]{\PP\left( #1 \right)}
\newcommand{\esp}[1]{\EE\left( #1 \right)}


%SystDist
\newcommand{\Receive}{\texttt{Receive~}}
\newcommand{\Send}{\texttt{Send~}}


%Preuves
\newcommand{\betaeq}{=_\beta}
\newcommand{\betared}{\vartriangleright_\beta}
\newcommand{\parabetared}{\vartriangleright_{||\beta}}
\newcommand{\Ackermann}{\A}


%Cplx
\newcommand{\Time}{\textsc{Time}}
\newcommand{\TIME}{\textsc{Time}}

\newcommand{\dtime}{\textsc{DTime}}
\newcommand{\dTime}{\textsc{DTime}}
\newcommand{\DTime}{\textsc{DTime}}

\newcommand{\ntime}{\textsc{NTime}}
\newcommand{\nTime}{\textsc{NTime}}
\newcommand{\NTime}{\textsc{NTime}}

\renewcommand{\P}{\textsc{P}}

\newcommand{\pTime}{\textsc{PTime}}
\newcommand{\PTime}{\textsc{PTime}}

\newcommand{\NP}{\textsc{NP}}

\newcommand{\npTime}{\textsc{NPTime}}
\newcommand{\NPTime}{\textsc{NPTime}}

\newcommand{\EXP}{\textsc{Exp}}
\newcommand{\expTime}{\textsc{Exp}}
\newcommand{\ExpTime}{\textsc{Exp}}
\newcommand{\EXPTime}{\textsc{Exp}}

\newcommand{\Space}{\textsc{Space}}

\newcommand{\dSpace}{\textsc{DSpace}}
\newcommand{\DSpace}{\textsc{DSpace}}


\newcommand{\nSpace}{\textsc{NSpace}}\newcommand{\NSpace}{\textsc{NSpace}}

\newcommand{\pSpace}{\textsc{PSpace}}
\newcommand{\PSpace}{\textsc{PSpace}}

\newcommand{\npSpace}{\textsc{NPSpace}}
\newcommand{\NpSpace}{\textsc{NPSpace}}
\newcommand{\NPSpace}{\textsc{NPSpace}}

\newcommand{\SpaceTM}{\textsc{SpaceTM}}

\newcommand{\nL}{\textsc{NL}}
\newcommand{\NL}{\textsc{NL}}

\newcommand{\LL}{\textsc{L}}

\newcommand{\coNP}{co\text{-}\textsc{NP}}

\newcommand{\conL}{co\text{-}\textsc{NL}}
\newcommand{\coNL}{co\text{-}\textsc{NL}}

\newcommand{\npc}{\text{\textit{NP-C}}}

\newcommand{\PH}{\textsc{PH}}

\newcommand{\TISP}{\textsc{TISP}}

\newcommand{\Size}{\textsc{Size}}
\newcommand{\SIZE}{\textsc{Size}}





\title{Calcul formel}
\author{
    Marc \textsc{Chevalier}
}
\date{\today}

\begin{document}
\maketitle

\section*{Produit court de polynômes}

\subsection*{Question 1}

On a $M\left(\frac{n}{2}\right) = C_\alpha n^\alpha$ avec $\alpha>1$.

La formule de récurrence est 

\[
    S(n) = 2S\left(\frac{n}{2}\right) + M\left(\frac{n}{2}\right)
\]

ie 

\[
    S(n) = 2S\left(\frac{n}{2}\right) + \frac{C_\alpha}{2^\alpha}n^\alpha
\]

En déroulant la récurrence:
\[
    S(n)=\frac{C_\alpha n^\alpha(2^{((\alpha-1)\log_2(n))}-1)2^{(\alpha-1)}}{2^\alpha(2^{(\alpha-1)}-1)2^{((\alpha-1)\log_2(n))}}
\]
soit
\[
    S(n) = \frac{C_\alpha(n^\alpha-n)}{2^\alpha-2}
\]

On a donc $S(n) = \Theta(n^\alpha)$. Ce qui n'est pas tout à fait assez fin mais qui montre que l'approche récursive ne sera pas nettement plus efficace que le calcul brutal.

\begin{enumerate}
    \item Dans le cas du produit avec l'algorithme naïf, on a $\alpha = 2$. Soit, dans notre cas
\[
    \begin{aligned}
        S(n) &= \frac{C_\alpha(n^2-n)}{2}\\
        &= \frac{C_2 n^2}{2} - \frac{C_2 n}{2}\\
        &= \frac{M(n)}{2} - \frac{C_2 n}{2}
    \end{aligned}
\]
    On a donc une légère amélioration par rapport à l'algorithme naïf.
    
    \item Dans le cas du produit avec l'algorithme de \textsc{Karatsuba}, on a $\alpha = \log_2(3)$. Soit, dans notre cas
\[
    \begin{aligned}
        S(n) &= \frac{C_{\log_2(3)}(n^{\log_2(3)}-n)}{2^{\log_2(3)}-2}\\
        &= \frac{C_{\log_2(3)}(n^{\log_2(3)}-n)}{3-2}\\
        &= C_{\log_2(3)}n^{\log_2(3)}-C_{\log_2(3)}n\\
        &= M(n) -C_{\log_2(3)} n
    \end{aligned}
\]
    On a donc une encore plus légère amélioration par rapport à l'algorithme de \textsc{Karatsuba}.
\end{enumerate}

\subsection*{Question 2}

En déroulant la récurrence, on trouve, puisque $M(kn) = k^\alpha M(n)$

\[
    S(n) = M(n) \beta^\alpha \frac{1-(2(1-\beta)^\alpha)^{\log_{\frac{1}{1-\beta}} (n)}}{1-2(1-\beta)^\alpha}
\]

On peut donc prendre $C_\beta = \frac{\beta^\alpha}{1-(2(1-\beta))^\alpha}$.

\subsection*{Question 3}

Il faut trouver le $\beta$ qui minimise $C_\beta$ en fonction de $\alpha$.

Pour se faire, il faut trouver $\beta$ qui annule $\frac{\partial C_\beta}{\partial \beta}$.

On a 
\[
    \frac{\partial C_\beta}{\partial \beta} =  \frac{\beta^\alpha \alpha}{\beta(1-2(1-\beta)^\alpha)}-\frac{2\beta^\alpha(1-\beta)^\alpha \alpha}{(1-2(1-\beta)^\alpha)^2(1-\beta)}
\]
et
\[
    \frac{\partial C_\beta}{\partial \beta} = 0 \Leftrightarrow \beta = -e^{-\frac{\ln(2)}{\alpha-1}} +1
\]

\bigskip

Pour $\alpha =2$, on a $\beta = \frac{1}{2}$ et $C_\beta = \frac{1}{2}$. Comme dans la question 1...

Pour $\alpha = \log_2(3)$, on a $\beta = -e^{-\frac{\ln(2)}{\frac{\ln(3)}{\ln(2)}-1}}+1\approx 0.69$ et $C_\beta = \frac{\left ( -e^{-\frac{\ln(2)}{\frac{\ln(3)}{\ln(2)}-1}}+1\right)^{\frac{\ln(3)}{\ln(2)}}}{1-2\left ( -e^{-\frac{\ln(2)}{\frac{\ln(3)}{\ln(2)}-1}}\right)^{\frac{\ln(3)}{\ln(2)}}}\approx 0.81$.

Ici, on a une amélioration sensible par rapport à la question 1.

\subsection*{Question 4}

Le produit naïf est plus rapide que l'algorithme de \textsc{Karatsuba} tant que
\[
    C_2 n^2 < C_{\log_2(3)} n^{\log_2(3)}
\]
soit 
\[
    n^{2-\log_2(3)} < \frac{C_{\log_2(3)}}{C_2}
\]
ie
\[
    n < \sqrt[2-\log_2(3)]{\frac{C_{\log_2(3)}}{C_2}} \approx \sqrt[0.42]{\frac{C_{\log_2(3)}}{C_2}}
\]

De même avec le produit court
\[
    \frac{1}{2}C_2n^2 < \frac{\left ( -e^{-\frac{\ln(2)}{\frac{\ln(3)}{\ln(2)}-1}}+1\right)^{\frac{\ln(3)}{\ln(2)}}}{1-2\left ( -e^{-\frac{\ln(2)}{\frac{\ln(3)}{\ln(2)}-1}}\right)^{\frac{\ln(3)}{\ln(2)}}} C_{\log_2(3)}n^{\log_2(3)}
\]
soit
\[
    n < \sqrt[0.42]{2\frac{\left ( -e^{-\frac{\ln(2)}{\frac{\ln(3)}{\ln(2)}-1}}+1\right)^{\frac{\ln(3)}{\ln(2)}}}{1-2\left ( -e^{-\frac{\ln(2)}{\frac{\ln(3)}{\ln(2)}-1}}\right)^{\frac{\ln(3)}{\ln(2)}}}} \sqrt[0.42]{\frac{C_{\log_2(3)}}{C_2}} \approx 3.13 \sqrt[0.42]{\frac{C_{\log_2(3)}}{C_2}}
\]

Ce seuil est donc 3.13 fois supérieur dans le cas du produit court.

\end{document}

