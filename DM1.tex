\input{src/macros.tex}

\title{Calcul formel}
\author{
    Marc \textsc{Chevalier}
}
\date{\today}

\begin{document}
\maketitle

\section*{Produit court de polynômes}

\subsection*{Question 1}

On a $M\left(\frac{n}{2}\right) = C_\alpha n^\alpha$ avec $\alpha>1$.

La formule de récurrence est 

\[
    S(n) = 2S\left(\frac{n}{2}\right) + M\left(\frac{n}{2}\right)
\]

ie 

\[
    S(n) = 2S\left(\frac{n}{2}\right) + \frac{C_\alpha}{2^\alpha}n^\alpha
\]

En déroulant la récurrence:
\[
    S(n)=\frac{C_\alpha n^\alpha(2^{((\alpha-1)\log_2(n))}-1)2^{(\alpha-1)}}{2^\alpha(2^{(\alpha-1)}-1)2^{((\alpha-1)\log_2(n))}}
\]
soit
\[
    S(n) = \frac{C_\alpha(n^\alpha-n)}{2^\alpha-2}
\]

On a donc $S(n) = \Theta(n^\alpha)$. Ce qui n'est pas tout à fait assez fin mais qui montre que l'approche récursive ne sera pas nettement plus efficace que le calcul brutal.

\begin{enumerate}
    \item Dans le cas du produit avec l'algorithme naïf, on a $\alpha = 2$. Soit, dans notre cas
\[
    \begin{aligned}
        S(n) &= \frac{C_\alpha(n^2-n)}{2}\\
        &= \frac{C_2 n^2}{2} - \frac{C_2 n}{2}\\
        &= \frac{M(n)}{2} - \frac{C_2 n}{2}
    \end{aligned}
\]
    On a donc une légère amélioration par rapport à l'algorithme naïf.
    
    \item Dans le cas du produit avec l'algorithme de \textsc{Karatsuba}, on a $\alpha = \log_2(3)$. Soit, dans notre cas
\[
    \begin{aligned}
        S(n) &= \frac{C_{\log_2(3)}(n^{\log_2(3)}-n)}{2^{\log_2(3)}-2}\\
        &= \frac{C_{\log_2(3)}(n^{\log_2(3)}-n)}{3-2}\\
        &= C_{\log_2(3)}n^{\log_2(3)}-C_{\log_2(3)}n\\
        &= M(n) -C_{\log_2(3)} n
    \end{aligned}
\]
    On a donc une encore plus légère amélioration par rapport à l'algorithme de \textsc{Karatsuba}.
\end{enumerate}

\subsection*{Question 2}

En déroulant la récurrence, on trouve, puisque $M(kn) = k^\alpha M(n)$

\[
    S(n) = M(n) \beta^\alpha \frac{1-(2(1-\beta)^\alpha)^{\log_{\frac{\beta - 1}{\beta}} (n)}}{1-2(1-\beta)^\alpha}
\]

On peut donc prendre $C_\beta = \frac{\beta^\alpha}{1-(2(1-\beta))^\alpha}$.

\subsection*{Question 3}

Il faut trouver le $\beta$ qui minimise $C_\beta$ en fonction de $\alpha$.

Pour se faire, il faut trouver $\beta$ qui annule $\frac{\partial C_\beta}{\partial \beta}$.

On a 
\[
    \frac{\partial C_\beta}{\partial \beta} =  \frac{\beta^\alpha \alpha}{\beta(1-2(1-\beta)^\alpha)}-\frac{2\beta^\alpha(1-\beta)^\alpha \alpha}{((1-2(1-\beta)^\alpha)^2(1-\beta)}
\]
et
\[
    \frac{\partial C_\beta}{\partial \beta} = 0 \Leftrightarrow \beta = -e^{-\frac{\ln(2)}{\alpha-1}} +1
\]

\bigskip

Pour $\alpha =2$, on a $\beta = \frac{1}{2}$ et $C_\beta = \frac{1}{2}$. Comme dans la question 1...

Pour $\alpha = \log_2(3)$, on a $\beta = -e^{-\frac{\ln(2)}{\frac{\ln(3)}{\ln(2)}-1}}+1\approx 0.69$ et $C_\beta = \frac{\left ( -e^{-\frac{\ln(2)}{\frac{\ln(3)}{\ln(2)}-1}}+1\right)^{\frac{\ln(3)}{\ln(2)}}}{1-2\left ( -e^{-\frac{\ln(2)}{\frac{\ln(3)}{\ln(2)}-1}}\right)^{\frac{\ln(3)}{\ln(2)}}}\approx 0.81$.

Ici, on a une amélioration sensible par rapport à la question 1.

\subsection*{Question 4}

Le produit naïf est plus rapide que l'algorithme de \textsc{Karatsuba} tant que
\[
    C_2 n^2 < C_{\log_2(3)} n^{\log_2(3)}
\]
soit 
\[
    n^{2-\log_2(3)} < \frac{C_{\log_2(3)}}{C_2}
\]
ie
\[
    n < \sqrt[2-\log_2(3)]{\frac{C_{\log_2(3)}}{C_2}} \approx \sqrt[0.42]{\frac{C_{\log_2(3)}}{C_2}}
\]

De même avec le produit court
\[
    \frac{1}{2}C_2n^2 < \frac{\left ( -e^{-\frac{\ln(2)}{\frac{\ln(3)}{\ln(2)}-1}}+1\right)^{\frac{\ln(3)}{\ln(2)}}}{1-2\left ( -e^{-\frac{\ln(2)}{\frac{\ln(3)}{\ln(2)}-1}}\right)^{\frac{\ln(3)}{\ln(2)}}} C_{\log_2(3)}n^{\log_2(3)}
\]
soit
\[
    n < \sqrt[0.42]{2\frac{\left ( -e^{-\frac{\ln(2)}{\frac{\ln(3)}{\ln(2)}-1}}+1\right)^{\frac{\ln(3)}{\ln(2)}}}{1-2\left ( -e^{-\frac{\ln(2)}{\frac{\ln(3)}{\ln(2)}-1}}\right)^{\frac{\ln(3)}{\ln(2)}}}} \sqrt[0.42]{\frac{C_{\log_2(3)}}{C_2}} \approx 3.13 \sqrt[0.42]{\frac{C_{\log_2(3)}}{C_2}}
\]

Ce seuil est donc 3.13 fois supérieur dans le cas du produit court.

\end{document}

